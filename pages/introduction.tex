\chapter{Introduction}\label{chap:introduction}

	Dans le cadre de mon stage de fin d'études, j'ai travaillé au sein d'IMT
	Atlantique, une école de l'Institut Mines-Télécom, sur un projet de recherche
	visant à compresser les structures de données de type Blockchain. Le stage,
	d'une durée de 6 mois, a commencé le 5 février 2024 et se terminera le 2 août
	2024. Il s'est déroulé au sein du département d'enseignement et de recherche
	"Systèmes Réseaux, Cybersécurité et Droit du numérique" (SRCD) d'IMT
	Atlantique, situé à Rennes.

	Le projet de recherche sur lequel j'ai travaillé avait pour objectif de
	proposer une solution pour compresser les structures de données de type
	Blockchain, et ainsi faire face aux problématiques de scalabilité que ces
	structures posent. Une telle solution existait déjà, mais elle fonctionnait
	dans un cadre très particulier : le cas où la population du système reste
	constante. L'objectif était de généraliser cette solution pour qu'elle
	fonctionne dans un cadre plus large, utilisable dans le monde réel, où la
	population du système est susceptible varier. J'ai intégré une équipe de 3
	chercheurs qui travaillaient déjà sur ce projet depuis quelques mois. J'ai
	apporté ma contribution à la solution proposée.
	 
	Mon mémoire de stage se décompose en trois chapitres. Le premier est la mise
	en contexte avec une présentation de mon environnement de travail, et du
	projet plus global dans lequel s'est inscrit mon stage. Le deuxième chapitre
	est une revue de l'état de l'art sur les systèmes Blockchain et sur
	l'algorithme de compression que nous avons utilisé. Enfin, le troisième
	chapitre est une présentation de mon travail, des problématiques que j'ai
	rencontrées, et des solutions que j'ai apportées.