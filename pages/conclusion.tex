\chapter{Conclusion}\label{chap:Conclusion}

    Au terme de ce stage, j'ai acquis de sérieuses connaissances et compétences
    dans le domaine des technologies Blockchains. Ce stage demandait plus de
    compétences en mathématiques que prévu, et j'ai fait des progrès
    significatifs dans ce domaine. En choisissant ce stage, j'avais comme
    objectif de découvrir le monde de la recherche. J'ai pu travailler avec des
    chercheurs d'IMT Atlantique, et j'ai pu découvrir le fonctionnement d'un
    laboratoire de recherche. J'ai également participé à des réunions de travail
    et à des séminaires. En somme, ce stage a été une expérience très
    enrichissante.

    Bien que l'article de recherche ne soit pas encore publié, on a pu obtenir
    des résultats, et il ne reste que quelques preuves et démonstrations à
    rédiger avant de pouvoir soumettre l'article à une conférence.    
    Plusieurs pistes d'amélioration sont envisageables pour la suite. On
    pourrait par exemple essayer d'adapter l'algorithme de compression à
    d'autres types de structures de données, comme les structures DAG (Directed
    Acyclic Graph) comme dans l'article~\cite{dag}. On pourrait aussi essayer
    d'affiner la valeur de $m$, en prenant en compte les effets du changement de
    target entre deux epoch. Pour conclure, je suis heureux d'avoir pu
    participer à ce projet.

    Pour finir ce mémoire, je tiens à remercier toutes les personnes qui m'ont
    aidé et soutenu durant ce stage. Je tiens à remercier mon tuteur et toute
    l'équipe de recherche d'IMT Atlantique pour m'avoir accueilli et encadré
    durant ces 6 mois.